\documentclass[conference]{IEEEtran}
%\IEEEoverridecommandlockouts
% The preceding line is only needed to identify funding in the first footnote. If that is unneeded, please comment it out.
%Template version as of 6/27/2024

\usepackage{cite}
\usepackage{float}
\usepackage{amsmath,amssymb,amsfonts}
\usepackage{algorithmic}
\usepackage{graphicx}
\usepackage{textcomp}
\usepackage{hyperref}

\usepackage{xcolor}
\def\BibTeX{{\rm B\kern-.05em{\sc i\kern-.025em b}\kern-.08em
    T\kern-.1667em\lower.7ex\hbox{E}\kern-.125emX}}

% comment commands you can change these colors
\newcommand{\linn}[1]{\textcolor{magenta}{(Linn:~#1)}}
\newcommand{\liya}[1]{\textcolor{orange}{(Liya:~#1)}} 
\newcommand{\jonathan}[1]{\textcolor{blue}{(Jonathan:~#1)}}

\bibliographystyle{IEEEtran}



\begin{document}

\title{Exploring a Pokemon dataset}

\author{\IEEEauthorblockN{1\textsuperscript{st} Linn Erle Kloefta}
\IEEEauthorblockA{\textit{Department of Computer Science} \\
\textit{Georgia State University}\\
Atlanta, United States \\
lklfta1@student.gsu.edu}
\and
\IEEEauthorblockN{2\textsuperscript{nd} Liya Sojan}
\IEEEauthorblockA{\textit{Department of Computer Science} \\
\textit{Georgia State University}\\
Atlanta, United States \\
lsojan1@student.gsu.edu}
\and
\IEEEauthorblockN{3\textsuperscript{rd} Jonathan Morales}
\IEEEauthorblockA{\textit{Department of Computer Science} \\
\textit{Georgia State University}\\
Atlanta, United States \\
jmorales32@student.gsu.edu}
}

\maketitle

\begin{abstract}
This paper analyzes a Pokemon dataset from Kaggle containing 1382 entries to explore three research questions: which types are most common and how single- versus dual-type Pokemon differ, which stats determine legendary status, and whether base stats can predict a Pokemon's type. We apply data mining techniques including exploratory data analysis, unsupervised clustering, correlation analysis, and supervised classification using logistic regression and random forest models. Our analysis reveals patterns in type distributions, identifies feature importance for classification tasks, and demonstrates the application of machine learning methods to gaming datasets with implications for understanding game mechanics and balance.
\end{abstract}
\vspace{0.2cm}


\begin{IEEEkeywords}
Pokemon, Data Exploration, Correlation, Clustering, Classification Models
\end{IEEEkeywords}


\section{Introduction}
\subsection{Motivation \& Background}
Growing up we all interacted with Pokemon either through the games or the TV-series, this led us to want to learn more about what makes up each Pokemon. Diving deeper into what each stat influences, how they are distributed across Pokemon with the same primary type, and investigating how the stats differ between legendary and normal Pokemon, seemed like an interesting challenge that inspired us to choose to explore this Pokemon dataset. 

In this report we will first look at the distribution of the data across the types of Pokemon, then we perform a correlational analysis looking at how the stats relate to each other and the target values, after that we examine the Pokemon by clustering and identifying natural groups, lastly we use the information we gained to train two different models to classify whether a Pokemon is legendary, and to classify the type of the Pokemon it is. 

Access the notebooks used during the analysis and modeling of our project on GitHub: \url{https://github.com/Redseba/Team-Rocket-Research}.

\subsection{Research Questions}

To tackle the data exploration we defined three key questions we wanted to answer. 
\subsubsection{Which types of Pokemon are most common, and is there a difference between single- and dual-type Pokemon}

We want to explore the types of Pokemon that exist, and what the breakdown looks like. And if there is anything interesting we did not know that stands out based on the type.

\subsubsection{Which stats are the most determining on whether a Pokemon is legendary or not}

Finding out what stats determine whether a Pokemon is classified as a legendary Pokemon can be valuable because it helps us narrow down what stats to look at when it comes to importance. It can also tell us things about the mechanics of the game that we did not know previously. 

\subsubsection{Can we use the base stats of a Pokemon to determine its type}

Are the six base stats; attack, defense, speed, health, special attack, and special defense, the stats that are most influential on what type category it gets classified as? If not, then what stats are, and is it possible to classify Pokemon with the correct types using data?

Together, these three research questions outline how we will explore the data in this report. 

\section{Materials and Methods}

\subsection{Data explanation and characterization}

The original dataset was retrieved from Kaggle\cite{kaggle_dataset}. It contained 45 columns, and had 1382 rows, with each row representing a Pokemon, and each column representing a stat. The stats cover information like types, abilities, egg groups, evolutions, attack and defense. 14 columns that had no useful data, such as ID's, game titles, and descriptions, were dropped. Shown in Figure~\ref{fig:distr_raw_pie} this left us with a total of 68.9\% of the original dataset. 

\begin{figure}[H]
    \centering
    \includegraphics[width=0.8\linewidth]{figures/distribution/raw_columns_pie.png}
    \caption{Overview of columns kept and excluded from the raw dataset.}
    \label{fig:distr_raw_pie}
\end{figure}

\subsection{Data Preprocessing}

When investigating the data we noticed that it contained some duplicate entries of Pokemon, due to descriptions changing upon newer releases of the games being released. After removing these duplicate Pokemon we were left with 1025 rows, which cover every Pokemon released in all games up until November 2022. 

\begin{figure}[H]
    \centering
    \includegraphics[width=0.8\linewidth]{figures/distribution/cleaned_columns_pie.png}
    \caption{Overview of original vs computed columns in the final cleaned dataset.}
    \label{fig:distr_cleaned_pie}
\end{figure}

To potentially help with analysis we computed and created 12 new columns, like BMI, which was calculated using the weight and height columns in the original dataset. After all the new columns were added to the dataset we were left with 42 columns, consisting of 72.1\% from the original dataset, and 27.9\% newly computed columns as shown in Figure~\ref{fig:distr_cleaned_pie}.

\subsection{Data analysis/mining}
Following our three defined research questions, we analyze the data for useful and insightful information. 
\subsubsection{Which types of Pokemon are most common, and is there
a difference between single- and dual-type Pokemon}

\begin{figure}[H]
    \centering
    \includegraphics[width=0.9\linewidth]{figures/distribution/single_vs_dual_pie.png}
    \caption{Overview of Pokemon that have single- vs dual-type.}
    \label{fig:distr_type_pie}
\end{figure}

To better understand the composition of our dataset, we have broken down the way Pokemon are classified through typing. Within the Pokemon universe, each Pokemon is assigned either only 1 elemental type (single-type) or 2 elemental types (dual-type). A single-type Pokemon can only possess one elemental type such as Fire, Water, or Grass, to name a few; as a result most Pokemon with single typing have very straightforward type match-ups (strengths and counters). In the case of dual-type Pokemon, they possess 2 elemental types leading to type combinations such as Fire/Flying, or Water/Ground, creating more complex interactions due to they inherit both strengths and weakness of their combined types. As shown in Figure~\ref{fig:distr_type_pie}, our data set reveals that 54.6\% of Pokemon are dual types, while the remaining 45.4\% are single types, showing a nearly even split in Pokemon type classifications with dual type Pokemon being slightly more common.

\begin{figure*}[htbp]
    \centering
    \includegraphics[width=0.9\linewidth]{figures/distribution/primary_type_all_and_single.png}
    \caption{Number of Pokemon with each primary type, for all Pokemon, or single-type only Pokemon.}
    \label{fig:distr_typeAllSingle_bar}
\end{figure*}

\begin{figure*}[htbp]
    \centering
    \includegraphics[width=0.9\linewidth]{figures/distribution/primary_and_secondary_dual_type.png}
    \caption{Number of Pokemon with primary and secondary type for dual-type Pokemon.}
    \label{fig:distr_typeDualSecondary_bar}
\end{figure*}

In Figure \ref{fig:distr_typeAllSingle_bar}, we observe that Water is the most common primary type overall with more than 100 Pokemon, and Flying the least common with only 9 Pokemon having it as their primary type. Interestingly we see that when we split the Pokemon up based on whether they are single- vs. dual-typed the top primary type differ between the groupings. Single-type Pokemon has Water type as the most common primary type, while in Figure~\ref{fig:distr_typeDualSecondary_bar} we can see that Bug type rose to the top as the most common primary type. This is an interesting observation, because it tells us that Bug type Pokemon are common, however they usually have more than one type. 

We also notice a clear reason for Flying being low in numbers when it comes to primary-types. In Figure~\ref{fig:distr_typeDualSecondary_bar} we see that Flying clearly stands out as the most common secondary-type for dual-type Pokemon with 100 Pokemon. This explains the absence of Flying type when divided by just the primary type. 

When investigating type pairings we see in Figure~\ref{fig:distr_top_combos_bar} that the most frequent ones are Normal \& Flying (25), Bug \& Flying (14), and Grass \& Poison (14), revealing that certain Pokemon typings are favored in design. It follows the expected pattern of having Flying as the most common secondary type, however it is an interesting deviation that Water is not among the three most frequent type combinations. This might be due to dual-type Water Pokemon having a greater variety in the type combinations, compared to Pokemon like Normal \& Flying which seem like a common pairing. The nearly even split shown in Figure~\ref{fig:distr_type_pie} highlights the strategic diversity of the franchise, with a lean towards dual-type Pokemon offering more complexity in design and battle mechanics through their type combinations.

\begin{figure}[H]
    \centering
    \includegraphics[width=1\linewidth]{figures/distribution/top_type_combos.png}
    \caption{Top 12 most common dual-type combinations among Pokemon.}
    \label{fig:distr_top_combos_bar}
\end{figure}

These type distribution patterns show that Pokemon follows systematic/game-design principles rather than assigning Pokemon typings at random, with certain combinations seeming to point to being heavily favored to create strategic gameplay progression and diversity

\subsubsection{Which stats are the most determining on whether a
Pokemon is legendary or not}

After examining type distributions, we now move onto analyzing the statistical attributes that define each Pokemon's capabilities in combat. The six base stats consist of: Health Points (\texttt{hp}), Attack (\texttt{atk}), Defense (\texttt{def}), Special Attack (\texttt{sp. atk}), Special Defense (\texttt{sp. def}), and \texttt{speed}. These stats determine a Pokemon's battle performance and often correlate with a Pokemon's overall role. Examining the stat distributions across all Pokemon and by their primary type, we can identify patterns that can reveal whether certain Pokemon types excel in certain stats, in addition to helping us understand what makes a Pokemon legendary either through stat distribution or status alone.

In Figure~\ref{fig:distr_stats_all_bar} we have visualized the distribution of the six base stats across all Pokemon types combined. This gives us a good overview of the ranges, and potential outliers within each stat category. Including covering the base stat distribution for all the Pokemon types combined, we also broke it down for each type which we can see in Figures \ref{fig:appendix_water} to \ref{fig:appendix_flying} in Appendix~\ref{sec:appendixA}. This allows us to look at how the stats are distributed within each type looking for potential patterns. We notice clear outliers in all categories which indicate that there are Pokemon present that lie outside the normal expected range. This is likely due to legendary Pokemon, and their unique stat distributions. 

\begin{figure*}[htbp]
    \centering
    \includegraphics[width=0.9\linewidth]{figures/distribution/base_stats_distributions.png}
    \caption{Distribution six base stats across all Pokemon types combined.}
    \label{fig:distr_stats_all_bar}
\end{figure*}


Comparing the mean of the base stats between normal and legendary Pokemon in Figure~\ref{fig:distr_legendary_normal_bar} we see that the averages for legendary Pokemon is clearly skewing more to the right than the averages for normal Pokemon. This tells us that legendary Pokemon tend to have stats that are on average higher than normal Pokemon. Although this is often the case, we still observe some clear outliers both in the high- and low-ranges of the base stat totals of the legendary Pokemon. 

Investigating these outliers in Figure~\ref{fig:distr_outlier_legendary} we see that the two Pokemon Cosmog and Eternatus lie at the very bottom, and top of the stat distribution. Cosmog has a lot lower base stats than the average legendary, and looking at Figure~\ref{fig:distr_legendary_normal_bar} it is also lower than most normal Pokemon. This led us to investigate whether the Pokemon could have been potentially mislabeled, however it is classified correctly as a legendary Pokemon according to Bulbapedia\cite{bulbapedia_cosmog} which is the wiki for the franchise. This tells us that there has to be something other than just the base stats that define a Pokemon as legendary, since those with both low and high stats can be classified as such. In the comparison we also looked at Eternatus which is the Pokemon that had higher than average legendary stats. As we can see in Figure~\ref{fig:distr_outlier_legendary} it has some stats like \texttt{hp, def},\texttt{sp. def} that are above average, and also stats that are within the expected range like \texttt{atk, sp.atk}, and \texttt{speed}. This shows that although some Pokemon have stats that can be seen as outliers, they might have some stats that lie in the normal range. This means that Eternatus is a specialist with higher stats in defensive categories, and lower stats in offensive categories to balance its average out. 

\begin{figure*}[htbp]
    \centering
    \includegraphics[width=0.9\linewidth]{figures/distribution/legendary_vs_normal_comparison.png}
    \caption{Distribution of six base stats comparing normal and legendary Pokemon.}
    \label{fig:distr_legendary_normal_bar}
\end{figure*}

\begin{figure*}[htbp]
    \centering
    \includegraphics[width=0.9\linewidth]{figures/distribution/legendary_lowest_highest_outlier_analysis.png}
    \caption{Outlier analysis of legendary Pokemon with high and low base stats.}
    \label{fig:distr_outlier_legendary}
\end{figure*}


After investigating the potential outliers, we decided that even though there are stats that are outside the expected normal, the Pokemon include unique, or important information that needs to be included. Excluding these Pokemon might hurt the model in terms of losing potentially important information. 

Other than looking at the distribution of stats, we performed a correlation analysis both between the stats themselves, as well as between the stats and the target values. The first target value is whether the Pokemon is legendary or not, and the second target value is the Pokemon's primary type. 

\begin{figure*}[htbp]
    \centering
    \includegraphics[width=0.9\linewidth]{figures/correlation/legendary_type_feature_correlations.png}
    \caption{Correlation analysis between features and targets.}
    \label{fig:corr_target}
\end{figure*}


In Figure~\ref{fig:corr_target} we investigate the fifteen most correlated features to the target values for both the models. For the legendary classifier we see that features like \texttt{has hidden ability}, \texttt{male and female ratio}, \texttt{base stat total}, and \texttt{ev yield total} have the highest numbers and therefore influence the target label most. Legendary Pokemon more often have hidden abilities compared to normal Pokemon, they also often consist of highly skewed male to female ratios which is likely why those are standing out. The \texttt{base stat total} also makes sense here because of the average being higher for legendary than normal Pokemon. The same applies to \texttt{ev yield total} which are hidden points that apply to the base stats. It also includes other stats like \texttt{base happiness} and \texttt{catch rate} which lets us know that legendary Pokemon might have abnormal base happiness rates, as well as lower catch rates than normal Pokemon. This makes sense as catching legendary Pokemon are significantly harder, and the low catch rates reflect that. 


\subsubsection{Can we use the base stats of a Pokemon to determine its type}

When directly investigating the distribution of the base stats of the Pokemon for the different types in Figures \ref{fig:appendix_water} to \ref{fig:appendix_flying} in Appendix~\ref{sec:appendixA} there are no clear patterns that stand out in terms of one type having certain stats average higher or lower. In Figure~\ref{fig:corr_target} the top features differ when we look at the primary type as the target compared to the legendary target, with features like \texttt{health ev}, \texttt{attack ev}, \texttt{physical defense}, \texttt{defense stat}, and \texttt{has secondary type} being the most influential on the primary type of a Pokemon. It does not include stats like \texttt{base stat total} or \texttt{catch rate} which leads us to think that the total base stats does not determine the type of a Pokemon, as well as its catch rate also not being influential here. This is interesting because it leads us to believe that the base stats might not be that influential on classifying the Pokemon into primary types. Instead it indicates that the ev's, which are hidden values that influence the base stats, correlate more with the primary type of a Pokemon. Whether a Pokemon has a secondary type also determines what kind of primary type it might have. This makes sense because of the different distributions of types between single-type and dual-type Pokemon.

To further investigate how the different features correlate we created a heatmap. All the numeric features are categorized into six groups: base stats, EV yields, physical attributes, derived statistics, categorical numeric features, and binary features. This allows us to calculate the correlation coefficients between all numeric features, legendary status, and encoded primary type. To do a correlation analysis of categorical variables, we used label encoding to convert the primary type and secondary type into numeric values. The heatmap then shows the pairwise correlations between forty features. Each cell in the heatmap represents the Pearson correlation coefficient between two features. The color range spans from dark red to dark blue, with dark red indicating a strong positive relationship and dark blue indicating a strong negative relationship. The diagonal on the heatmap shows a perfect correlation because each feature correlates with itself perfectly. Other strong positive correlations are shown in the base stats cluster in the upper-left corner. The base stats of Attack Stat, Defense Stat, and Special Attack Stat are moderately correlated with each other, because Pokemon's with high attack also tend to have high overall stats. We can see that the primary type encoded do not have a clear relationships with any stats, which might lead to the model having a very low accuracy score.


\begin{figure*}[htbp]
    \centering
    \includegraphics[width=0.7\linewidth]{figures/correlation/full_feature_correlation_heatmap.png}
    \caption{Heatmap showing relationships between Pokemon features}
    \label{fig:corr_heatmap}
\end{figure*}

To complement this correlation analysis and explore whether Pokemon naturally group by type based on their stats, we performed an unsupervised clustering analysis. If base stats were strongly predictive of type, we would expect Pokemon to cluster into type-specific groups. However, rather than using absolute stat values (which reflect power level), we clustered based on stat proportions and specialization metrics. This approach captures how stats are distributed (each stat as a percentage of total stats) and includes specialization features such as Physical vs Special and Offensive vs Defensive balance. By using proportions rather than absolute values, we remove power level bias and focus on combat role patterns, allowing both weak and strong Pokemon with similar battle strategies to group together.


Using the elbow method to select the optimal number of clusters, we identified k=8 as the appropriate number of cluster groups (Figure~\ref{fig:cluster_elbow} in Appendix~\ref{sec:appendixA}). The clustering achieved a silhouette score of 0.1886, indicating weak but detectable cluster structure. While this relatively low score suggests considerable overlap between combat role archetypes, it is consistent with the expectation that Pokemon stat distributions form a continuum rather than discrete categories. We then visualized these clusters using t-SNE dimensionality reduction in Figure~\ref{fig:cluster_tsne}, which shows separation between cluster groups while revealing areas of overlap.

\begin{figure}[H]
    \centering
    \includegraphics[width=1\linewidth]{figures/clustering/clusters_tsne.png}
    \caption{Cluster groups visualized using t-SNE dimensionality reduction.}
    \label{fig:cluster_tsne}
\end{figure}

When examining the type distribution within each cluster (Figure~\ref{fig:cluster_overview}in Appendix~\ref{sec:appendixA}), we found that clusters do not align with Pokemon types. All clusters contain multiple different types, with cluster sizes ranging from 25 Pokemon (2.4\%) to 179 Pokemon (17.5\%) but no cluster dominated by a single type. This provides strong evidence that stat distribution patterns do not correspond to type classification. This reinforces the findings from our correlation analysis: Pokemon types cannot be reliably predicted from base stats alone because different types share similar combat role archetypes.

Instead of type-based groups, the clustering revealed distinct combat role archetypes. Figure~\ref{fig:cluster_radar} in Appendix~\ref{sec:appendixA} presents radar charts showing each cluster's stat profile, with the dashed gray line representing the overall average for comparison. These profiles reveal archetypes such as glass cannons (high offensive stats with low defenses), generalists (balanced stat distributions), and tanks (emphasis on defensive stats over offense). The fact that these archetypes span multiple types demonstrates that a Water-type and a Fire-type Pokemon can have nearly identical stat distributions if they serve similar combat roles.


Figure~\ref{fig:cluster_deviation} in Appendix~\ref{sec:appendixA} quantifies how each cluster's average stats deviate from the overall mean, measured in standard deviations. Positive values indicate above-average stats for that cluster, while negative values indicate below-average stats. The magnitude of deviation reveals which stats most strongly define each archetype. For example, a cluster with high positive deviation in Attack and negative deviation in Defense represents an offensive-focused archetype that includes Pokemon of many different types.


Together, the correlation analysis and clustering results provide converging evidence: base stats reflect combat roles and battle strategies rather than type classification. The weak silhouette score (0.1886) and the presence of all types across all clusters both demonstrate that stat-based groupings are independent of type, explaining why type prediction models may struggle with low accuracy despite clear stat-based combat archetypes emerging from the data.



\subsection{Evaluation and interpretations}
After the exploratory data analysis, it led us to our modeling approach. Once we examined the type distributions and identified that the legendary Pokemon showed higher base stats through the correlation analysis, er developed two classification models to help us answer our research questions. There is a strong correlation between stats and legendary status which would suggest that it would a successful prediction task, while the lack of clear stat and type relationship indicates that type classification would be challenging.  With these insights, it guided our feature selection and model choice. We applied both Logistic Regression and Random Forest classifiers to compare linear versus ensemble approaches. 

\subsubsection{Legendary Classifier}
We decided to use k-fold cross-validation to make sure that our model generalizes well to unseen data and to prevent overfitting. Cross-validation will split the data into 'k' sized folds, train on k-1 folds, and test on the remaining one. This process will be repeated for all combinations, and will provide a better estimate of the model performance than a single train-test split. The confusion matrix will show details about model performance. For our Logistic Regression as seen in Figure~\ref{fig:LogReg}, it produced 23 true negatives, 3 false positives, 17 false negatives, and 214 true positives. Whereas Random Forest  as seen in Figure~\ref{fig:RandForest} produced 23 true negatives, 3 false positives, 1 false negative, and 230 true positives. Therefore, Random Forest reduced both types of errors, in particular false negatives were decreased from four to one.
\begin{figure}
    \centering
    \includegraphics[width=0.75\linewidth]{conf_mat_legendary_lr_tuned.png}
    \caption{Model predictions vs actual legendary status}
    \label{fig:LogReg}
\end{figure}

\begin{figure}
    \centering
    \includegraphics[width=0.75\linewidth]{conf_mat_legendary_rf_tuned.png}
    \caption{Model predictions vs actual legendary status}
    \label{fig:RandForest}
\end{figure}

We tuned the hyperparameters including the number of trees, maximum depth, and minimum samples per split, for the Random Forest model. F1 was selected as our primary evaluation metric for tuning because it will balance precision and recall. This is especially crucial when dealing with imbalanced data. To make sure that our model does not achieve a high accuracy by predicting "non-legendary" for everything (since legendary Pokemon are more rare than non-legendary ones), the F1 score will ensure that we are correctly identifying both classes.

We decided to select Random Forest as our final selection for many reasons. As seen in Table~\ref{tab:legendary} Random Forest has a higher F1 score (0.93) than Logistic Regression (0.79), which shows a better overall balance between precision and recall. Since the recall is also higher (0.96) than Logistic Regression (0.89), it will mean that there are less false alarms. Lastly, the ensemble approach better captures non-linear relationships in the data.

\subsubsection{Primary Type Classifier}
We decided to use the same k-fold cross-validation approach for the type classifier to make sure there is a consistent evaluation between both models. The confusion matrix from type classifier as seen Figure~\ref{fig:TypeLogReg} and Figure~\ref{fig:TypeRandForest} showed widespread misclassifications across all types. This would confirm that using base stats alone does not provide sufficient information for type prediction. There were no single type that showed a strong predictive accuracy, indicating systematic challenges rather than isolated errors. We tuned hyperparameters for both models and used a weighted F1 score as out metric, this would account for the class imbalance among the 18 types. Some types were more common than other, but weighted F1 will make sure that we are not optimizing for predicting the most frequent types. 

\begin{figure}
    \centering
    \includegraphics[width=0.75\linewidth]{confusion_matrix_lr_tuned.png}
    \caption{Confusion matrix showing logistic regression classification performance across Pokemon types}
    \label{fig:TypeLogReg}
\end{figure}

\begin{figure}
    \centering
    \includegraphics[width=0.75\linewidth]{confusion_matrix_rf_tuned.png}
    \caption{Confusion matrix showing random forest classification performance across Pokemon types}
    \label{fig:TypeRandForest}
\end{figure}

Even though Random Forest performed better, both models had a poor performance, which shows that the feature set of base stats is insufficient for this classification task.  Since both models had a similar performance, this would show that the problem lies in the features themselves instead of the model complexity, because even Random Forest could not extract meaningful patterns.

\section{Results}
Here we will talk about the results achieved from the two(four) models and show the associated summary tables

\subsection{Legendary Pokemon Classifier Results}
After incorporating hyperparameter tuning and cross-validation, as seen both models displayed strong performances in predicting a Pokemon's legendary status. As seen in Table~\ref{tab:legendary} our Logistic Regression model achieved 95\% accuracy with a F1 score of 0.79, this shows that even a linear model can separate legendary Pokemon from non-legendaries effectively. As also seen on Table~\ref{tab:legendary} our  Random Forest classifier outperformed the Logistic Regression across in all categories, achieving a 98\% accuracy, a 89\% in precision, and lastly a 96\% in recall in addition to having a F1 score of 0.93. With its superior performance it can be inferred that the Random Forest model better captures the complex, non-linear relationships between a Pokemon's stats and their status in being a legendary. Most particularly important to point out is the 96\% signifying that model correctly identifies nearly all legendary with very little false negatives

\begin{table}[h]
\centering
\caption{Legendary Pokémon Classifier Performance}
\begin{tabular}{lcccc}
\hline
\textbf{Model} & \textbf{Accuracy} & \textbf{Precision} & \textbf{Recall} & \textbf{F1} \\
\hline
Logistic Regression & 0.95 & 0.73 & 0.85 & 0.79 \\
Random Forest & 0.98 & 0.89 & 0.96 & 0.93 \\
\hline
\end{tabular}
\label{tab:legendary}
\end{table}

\subsection{Pokemon Type Classifier Results}
For the type classification, the task proved to be more challenging in comparison to the legendary classifier. In the case of the Logistic Regression model as seen in Table~\ref{tab:type} it achieved a 26\% accuracy with an F1 score of 0.21, while with the Random Forest model additionally locared in Table~\ref{tab:type} it showed only very marginal improvements with only a accuracy of 28\% and a F1 score of 0.24. Even though our numbers appeared low, it's important to take into consideration with a total of 18 primary types, random guessing would only have a accuracy of 5.6\%, from this it could be said our models performed roughly 5 times as better than guessing from chance. However, these performances indicate that base stats alone are not enough in regards to predicting a Pokemon's type, as mentioned type it seems typing is more driven based off design themes, elemental characteristics, and in-universe lore instead of pure statistical attributes.

\begin{table}[h]
\centering
\caption{Pokémon Type Classifier Performance}
\begin{tabular}{lcccc}
\hline
\textbf{Model} & \textbf{Accuracy} & \textbf{Precision} & \textbf{Recall} & \textbf{F1} \\
\hline
Logistic Regression & 0.26 & 0.21 & 0.23 & 0.21 \\
Random Forest & 0.28 & 0.25 & 0.25 & 0.24 \\
\hline
\end{tabular}
\label{tab:type}
\end{table}

The visible contrast between the two classifications yields an important insight: while a Pokemon's legendary status correlates heavily with a higher base stat total, Pokemon types reflect conceptual and thematic decisions that cannot be derived from numerical features alone.

\section{Discussion and Conclusion}
\subsection{Why the Legendary Classifier Succeeded}
Our legendary Pokemon classifier as seen in Table~\ref{tab:legendary} achieved highly accurate results with an accuracy of 98\% as a result of the type of classification problem it is containing clear patterns. Legendary Pokemon consistently have greater base stat total when in comparison to non-legendary Pokemon, in-turn creating a visible distinction that the model could easily identify. The Random Forest model performed best due to it being able to capture the complex relationships between the differing stats. This type of classification problem generally is much more feasible when compared to a problem that uses multiple classes since the model only has to separate 2 groups instead of many

\subsection{Why the Primary Type Classifier Performed Poorly}
The type classifier as seen on Table~\ref{tab:type} had a drastically different outcome in comparison to the legendary model. With an accuracy of 28\% the type classifier model displayed that Pokemon types don't strongly correlate with base stats. Types are assigned base on either themes or design choices, not heavily relying on statistical patterns. A Fire-type and Grass-type Pokemon might have the same stat totals, making them very hard to distinguish from one another using numbers alone. However there is a possibility of improving the accuracy of the classifier by using more richer data such as abilities, move pools, and etc.

\subsection{Research Questions Revisited}
\subsubsection{Which types of Pokemon are most common, and is there a difference between single and dual-type Pokemon}
The common Pokemon types consist of Water, Normal, and Grass types serving as the top 3. The main difference between dual-type and single type Pokemon, is that dual type Pokemon are more prevalent than single-type, with combinations like Water/Flying and Grass/Poison appearing very frequently.

\subsubsection{Which stats are the most determining on whether a Pokemon is legendary or not?}
The strongest indicators for whether a Pokemon is legendary or not are overall base stat totals, Special Attack, Special Defense, and HP. Our Random Forest model's 98\% accuracy confirms that these stats determine a Pokemon's legendary status

\subsubsection{Can we use the base stats of a Pokemon is legendary or not?}
No, this is due to the result of our accuracy being 28\% as seen on Table~\ref{type} determining base stats alone cannot predict a Pokemon's typing. A Pokemon's type assignment is drive beyond factors like statistical attributes.

\subsection{Unexpected Findings}
We were most surprised with how type classification resulted so poorly, we expected types to align with stat patterns (More defensive type Pokemon such as Rock or Steel to have higher defense, etc.), but instead our data showed otherwise. Alongside that we were impressed by how the legendary classifier came out, suggesting intentional game design when creating legendary Pokemon. 

\subsection{Future Work}
To improve upon our type classifications, we would need to add additional features such as move pools, abilities, and visual characteristics as mentioned previously. Our future research could extend into predicting competitive viability of Pokemon, usage rates in battles, and expanding upon our successful legendary classification framework to incorporate detections for pseudo-legendary or mythical Pokemon.

\bibliography{references}

\newpage
\appendices
\section{Additional Figures}
\label{sec:appendixA}


\begin{figure*}[htbp]
    \centering
    \includegraphics[width=0.9\linewidth]{figures/distribution/all_stats_Water_type.png}
    \caption{Distribution of base stats across all Water type Pokemon.}
    \label{fig:appendix_water}
\end{figure*}

\begin{figure*}[htbp]
    \centering
    \includegraphics[width=0.9\linewidth]{figures/distribution/all_stats_Normal_type.png}
    \caption{Distribution of base stats across all Normal type Pokemon.}
    \label{fig:placeholder}
\end{figure*}

\begin{figure*}[htbp]
    \centering
    \includegraphics[width=0.9\linewidth]{figures/distribution/all_stats_Grass_type.png}
    \caption{Distribution of base stats across all Grass type Pokemon.}
    \label{fig:placeholder}
\end{figure*}

\begin{figure*}[htbp]
    \centering
    \includegraphics[width=0.9\linewidth]{figures/distribution/all_stats_Bug_type.png}
    \caption{Distribution of base stats across all Bug type Pokemon.}
    \label{fig:placeholder}
\end{figure*}

\begin{figure*}[htbp]
    \centering
    \includegraphics[width=0.9\linewidth]{figures/distribution/all_stats_Psychic_type.png}
    \caption{Distribution of base stats across all Psychic type Pokemon.}
    \label{fig:placeholder}
\end{figure*}

\begin{figure*}[htbp]
    \centering
    \includegraphics[width=0.9\linewidth]{figures/distribution/all_stats_Fire_type.png}
    \caption{Distribution of base stats across all Fire type Pokemon.}
    \label{fig:placeholder}
\end{figure*}

\begin{figure*}[htbp]
    \centering
    \includegraphics[width=0.9\linewidth]{figures/distribution/all_stats_Electric_type.png}
    \caption{Distribution of base stats across all Electric type Pokemon.}
    \label{fig:placeholder}
\end{figure*}

\begin{figure*}[htbp]
    \centering
    \includegraphics[width=0.9\linewidth]{figures/distribution/all_stats_Rock_type.png}
    \caption{Distribution of base stats across all Rock type Pokemon.}
    \label{fig:placeholder}
\end{figure*}

\begin{figure*}[htbp]
    \centering
    \includegraphics[width=0.9\linewidth]{figures/distribution/all_stats_Dark_type.png}
    \caption{Distribution of base stats across all Dark type Pokemon.}
    \label{fig:placeholder}
\end{figure*}

\begin{figure*}[htbp]
    \centering
    \includegraphics[width=0.9\linewidth]{figures/distribution/all_stats_Poison_type.png}
    \caption{Distribution of base stats across all Poison type Pokemon.}
    \label{fig:placeholder}
\end{figure*}

\begin{figure*}[htbp]
    \centering
    \includegraphics[width=0.9\linewidth]{figures/distribution/all_stats_Fighting_type.png}
    \caption{Distribution of base stats across all Fighting type Pokemon.}
    \label{fig:placeholder}
\end{figure*}

\begin{figure*}[htbp]
    \centering
    \includegraphics[width=0.9\linewidth]{figures/distribution/all_stats_Steel_type.png}
    \caption{Distribution of base stats across all Steel type Pokemon.}
    \label{fig:placeholder}
\end{figure*}

\begin{figure*}[htbp]
    \centering
    \includegraphics[width=0.9\linewidth]{figures/distribution/all_stats_Ice_type.png}
    \caption{Distribution of base stats across all Ice type Pokemon.}
    \label{fig:placeholder}
\end{figure*}

\begin{figure*}[htbp]
    \centering
    \includegraphics[width=0.9\linewidth]{figures/distribution/all_stats_Ground_type.png}
    \caption{Distribution of base stats across all Ground type Pokemon.}
    \label{fig:placeholder}
\end{figure*}

\begin{figure*}[htbp]
    \centering
    \includegraphics[width=0.9\linewidth]{figures/distribution/all_stats_Ghost_type.png}
    \caption{Distribution of base stats across all Ghost type Pokemon.}
    \label{fig:placeholder}
\end{figure*}

\begin{figure*}[htbp]
    \centering
    \includegraphics[width=0.9\linewidth]{figures/distribution/all_stats_Dragon_type.png}
    \caption{Distribution of base stats across all Dragon type Pokemon.}
    \label{fig:placeholder}
\end{figure*}
\begin{figure*}[htbp]
    \centering
    \includegraphics[width=0.9\linewidth]{figures/distribution/all_stats_Fairy_type.png}
    \caption{Distribution of base stats across all Fairy type Pokemon.}
    \label{fig:placeholder}
\end{figure*}
\begin{figure*}[htbp]
    \centering
    \includegraphics[width=1\linewidth]{figures/distribution/all_stats_Flying_type.png}
    \caption{Distribution of base stats across all Flying type Pokemon.}
    \label{fig:appendix_flying}
\end{figure*}

\begin{figure*}[htbp]
    \centering
    \includegraphics[width=1\linewidth]{figures/clustering/elbow_method.png}
    \caption{Elbow method showing selection of optimal k clusters.}
    \label{fig:cluster_elbow}
\end{figure*}

\begin{figure*}[htbp]
    \centering
    \includegraphics[width=0.6\linewidth]{figures/clustering/type_distribution.png}
    \caption{Distribution of the five most common types in each cluster, showing that clusters are not type-specific.}
    \label{fig:cluster_overview}
\end{figure*}

\begin{figure*}[htbp]
    \centering
    \includegraphics[width=0.6\linewidth]{figures/clustering/cluster_radar_subplots.png}
    \caption{Radar plots showing the average stat profile for each cluster compared to the overall average (dashed line).}
    \label{fig:cluster_radar}
\end{figure*}

\begin{figure*}[htbp]
    \centering
    \includegraphics[width=1\linewidth]{figures/clustering/cluster_deviations.png}
    \caption{Deviation from overall mean (in standard deviations) for each cluster, showing which stats define each archetype. Positive values indicate above-average stats; negative values indicate below-average stats.}
    \label{fig:cluster_deviation}
\end{figure*}




\end{document}
