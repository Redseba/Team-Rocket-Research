\documentclass[conference]{IEEEtran}
%\IEEEoverridecommandlockouts
% The preceding line is only needed to identify funding in the first footnote. If that is unneeded, please comment it out.
%Template version as of 6/27/2024

\usepackage{cite}
\usepackage{float}
\usepackage{amsmath,amssymb,amsfonts}
\usepackage{algorithmic}
\usepackage{graphicx}
\usepackage{textcomp}
\usepackage{xcolor}
\def\BibTeX{{\rm B\kern-.05em{\sc i\kern-.025em b}\kern-.08em
    T\kern-.1667em\lower.7ex\hbox{E}\kern-.125emX}}

% comment commands you can change these colors
\newcommand{\linn}[1]{\textcolor{magenta}{(Linn:~#1)}}
\newcommand{\liya}[1]{\textcolor{orange}{(Liya:~#1)}} 
\newcommand{\jonathan}[1]{\textcolor{blue}{(Jonathan:~#1)}} 

\begin{document}

\title{Exploring a Pokemon dataset}

\author{\IEEEauthorblockN{1\textsuperscript{st} Given Name Surname}
\IEEEauthorblockA{\textit{Department of Computer Science} \\
\textit{Georgia State University}\\
Atlanta, United States \\
email address or ORCID}
\and
\IEEEauthorblockN{2\textsuperscript{nd} Given Name Surname}
\IEEEauthorblockA{\textit{Department of Computer Science} \\
\textit{Georgia State University}\\
Atlanta, United States \\
email address or ORCID}
\and
\IEEEauthorblockN{3\textsuperscript{rd} Given Name Surname}
\IEEEauthorblockA{\textit{Department of Computer Science} \\
\textit{Georgia State University}\\
Atlanta, United States \\
email address or ORCID}
}

\maketitle

\begin{abstract}
Briefly introduce your study , your analyses, your results and conclusion
\end{abstract}

\vspace{0.5cm}

\begin{IEEEkeywords}
keyword, keyword, keyword, keyword
\end{IEEEkeywords}

\vspace{0.5cm}

\section{Introduction}
\linn{To add comments in the text easily} \liya{double click me to go to the code and see how to use your named command} \jonathan{you can change the colors at the top} Background of the study. Why do you want to study this
problem. Why this problem is important or interesting? Etc.
What has been done about this problem? What do you want
to achieve? What is the uniqueness or the contribution of your
study? Etc. 

We should form 2 or 3 research questions we want to answer and then have one section later in the results discussing each. We should introduce each here first. 


Brief what is your study and report organization. Provide
the link to your code. 


\section{Materials and Methods}

\subsection{Data explanation and characterization}
What data are you using? where the data come from, data
type, data size, data range , etc

\begin{figure}[H]
    \centering
    \includegraphics[width=1\linewidth]{figures/distribution/raw_columns_pie.png}
    \caption{Overview of columns kept and excluded from the raw dataset.}
    \label{fig:distr_raw_pie}
\end{figure}

\subsection{Data preprocessing}
What you did on quality check, data clearing, data selection
or data transformation, etc.

\begin{figure}[H]
    \centering
    \includegraphics[width=1\linewidth]{figures/distribution/cleaned_columns_pie.png}
    \caption{Overview of original vs computed columns in the final cleaned dataset.}
    \label{fig:distr_cleaned_pie}
\end{figure}

\subsection{Data analysis/mining}
\subsubsection{Pokemon Type}

\begin{figure}[H]
    \centering
    \includegraphics[width=1\linewidth]{figures/distribution/single_vs_dual_pie.png}
    \caption{Overview of Pokemon that have single- vs dual-type.}
    \label{fig:distr_type_pie}
\end{figure}

\begin{figure*}[htbp]
    \centering
    \includegraphics[width=1\linewidth]{figures/distribution/primary_type_all_and_single.png}
    \caption{Number of Pokemon with each primary type, for all Pokemon, or single-type only Pokemon.}
    \label{fig:distr_typeAllSingle_bar}
\end{figure*}

\begin{figure*}[htbp]
    \centering
    \includegraphics[width=1\linewidth]{figures/distribution/primary_and_secondary_dual_type.png}
    \caption{Number of Pokemon with primary and secondary type for dual-type Pokemon.}
    \label{fig:distr_typeDualSecondary_bar}
\end{figure*}

\begin{figure}[H]
    \centering
    \includegraphics[width=1\linewidth]{figures/distribution/top_type_combos.png}
    \caption{Caption}
    \label{fig:placeholder}
\end{figure}

\subsubsection{Pokemon Stats}

\begin{figure*}[htbp]
    \centering
    \includegraphics[width=1\linewidth]{figures/distribution/base_stats_distributions.png}
    \caption{Caption}
    \label{fig:placeholder}
\end{figure*}



\subsection{Evaluation and interpretations}
Metrics used for evaluation such as confusion table,
significance level. Validation methods used such as cross
validation, independent test.

\begin{figure*}[htbp]
    \centering
    \includegraphics[width=1\linewidth]{figures/correlation/full_feature_correlation_heatmap.png}
    \caption{Caption}
    \label{fig:placeholder}
\end{figure*}

\section{Results}
Present your results. Here you can use tables and figures to
clearly show the results, which matches the analyses step-by
step


Using H on the figure lets me add writing below when it usually would automatically be moved below the other paragraph due to it having space. 

\begin{table}[H]
\caption{Caption of the table}
\begin{center}
\begin{tabular}{|c|c|c|c|}
\hline
\textbf{Table}&\multicolumn{3}{|c|}{\textbf{Table Column Head}} \\
\cline{2-4} 
\textbf{Head} & \textbf{\textit{Table column subhead}}& \textbf{\textit{Subhead}}& \textbf{\textit{Subhead}} \\
\hline
copy& More table copy$^{\mathrm{a}}$& &  \\
\hline
\multicolumn{4}{l}{$^{\mathrm{a}}$Sample of a Table footnote.}
\end{tabular}
\label{tab1}
\end{center}
\end{table}

Same goes for the table.

\section{Discussion and Conclusion}
Regarding the results you get, what are your understanding of
them, for example , the model is surprisingly good or bad, and
why; the results are as expected or not in terms relation,
important features, etc.
From the results, what you have learned and you want to let
your readers know.
Any limitation of your study, what further works can be done
to improve.
Overall summary of our study


\section*{Acknowledgment}

People who have helped your study but not in the author list.
Resources you have used like data source, funding recourses for the study

\begin{thebibliography}{00}
\bibitem{b1} G. Eason, B. Noble, and I. N. Sneddon, ``On certain integrals of Lipschitz-Hankel type involving products of Bessel functions,'' Phil. Trans. Roy. Soc. London, vol. A247, pp. 529--551, April 1955.
\end{thebibliography}

% \vspace{12pt}
% \color{red}
% IEEE conference templates contain guidance text for composing and formatting conference papers. Please ensure that all template text is removed from your conference paper prior to submission to the conference. Failure to remove the template text from your paper may result in your paper not being published.

\end{document}
